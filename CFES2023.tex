%%%%%%%%%%%%%%%%%%%%%%%%%%%%%%%%%%%%%%%%%%%%%%%%%%%%%%%%%%%%%%%%%%%%%%%%%%%%%%%%%%%%%%%%%%%%%%%%%%%
%
% Notes on Citations
%
% MooreAfter2017.bib is an export of the MooreAfter2017 saved search in my Zotero database.
%
% Make sure that the citekey is pinned to each item ( Right click; Better Bibtex | Pin Bibtex key)
%
% After uploading, click on Tools | Bibliography, then open the log file to see any warnings
% or errors which need to be fixed. 
%
% Fix any problems in Zotero, then export a fresh version of MooreAfter2017.bib
%
% Keywords in the MooreAfter2017.bib are used as filters. Note that keywords
% cannot contain space characters.
% 
% Example:
% \begin{refsection}
%	\nocite{*}
%	\printbibliography[heading=none, keyword={Moore-presentations-after-2017}]
% \end{refsection} 
%
%%%%%%%%%%%%%%%%%%%%%%%%%%%%%%%%%%%%%%%%%%%%%%%%%%%%%%%%%%%%%%%%%%%%%%%%%%%%%%%%%%%%%%%%%%%%%%%%%%%
%
% Checking links in PDF
%
% pdfx -c CFES2019.pdf
%
%%%%%%%%%%%%%%%%%%%%%%%%%%%%%%%%%%%%%%%%%%%%%%%%%%%%%%%%%%%%%%%%%%%%%%%%%%%%%%%%%%%%%%%%%%%%%%%%%%%
%
% Subsection format
%
% \begin{refsection}
% \subsection{Description}
% \subsection{Activities}
% \subsection{Plans}
% \subsection{References}
% \printbibliography[heading=none]
% \end{refsection}
%
%%%%%%%%%%%%%%%%%%%%%%%%%%%%%%%%%%%%%%%%%%%%%%%%%%%%%%%%%%%%%%%%%%%%%%%%%%%%%%%%%%%%%%%%%%%%%%%%%%%

\documentclass[12pt,english,letterpaper]{scrartcl}
\usepackage{fullpage}
\usepackage[T1]{fontenc}
\usepackage{color}
\usepackage{array}
\usepackage{url}
\usepackage{pdfpages}
\usepackage{longtable}
\usepackage{booktabs}
\usepackage[utf8]{inputenc}
\usepackage[english]{babel}
\usepackage{csquotes}
\usepackage{xcolor}
\usepackage{pgfgantt}
\usepackage{todonotes}
\usepackage{float}

% Use style=draft to print citation keys
% sorting=none orders references in the order in which they occur in the document
% maxbibnames=99 prevents use of et al.
%\usepackage[sorting=none, style=draft, maxbibnames=99]{biblatex}
\usepackage[sorting=none, maxbibnames=99]{biblatex}

% This code adds a category "cited" to cited bibliographic items
\DeclareBibliographyCategory{cited}
\AtEveryCitekey{\addtocategory{cited}{\thefield{entrykey}}}

% IMPORTANT NOTE: Zotero does not provide automatic sync for saved searches. This must be done manually.

\addbibresource{MooreAfter2017.bib}    % From the My Library in my Zotero
\addbibresource{MooreCRBAfter2019.bib} % From the CRB Library in my Zotero
\addbibresource{inat.bib}              % From ?
\addbibresource{misc.bib}              % manually editted
\addbibresource{InPreparation.bib}     % From my Zotero; saved search for "In preparation" in Extras
\addbibresource{refs.bib}              % references for GISC report

% Add all build a bibliography containing all references, cited plus uncited
%\nocite{*}

\usepackage[breaklinks=true, colorlinks=True, allcolors=blue]{hyperref}

\usepackage{indentfirst} 
\usepackage{comment}

% A couple of very simple macros to add 'Activities' and 'Plans' headings.
\newcommand{\activities}{\medskip\textbf{Activities}}
\newcommand{\plans}{\medskip\textbf{Plans}}

\makeatletter

\makeatother

\errorcontextlines=3 % For checking biblatex

\addtokomafont{section}{\color{blue}}
\addtokomafont{subsection}{\color{blue}}


\begin{document}

\title{CFES Report\\2022-06-26 to 2023-06-26}

\author{Aubrey Moore, Ph.D.\\
Professor / Extension Entomologist}

\maketitle

\begin{center}
	\href{https://aubreymoore.github.io/CFES2020-22/CFES2023.pdf}{Click here to download the latest version of this document.}
\end{center}


%\includepdf[pages=1]{Reflective-form}
\setcounter{secnumdepth}{0} % Prevents numbers added to TOC entries
\setcounter{tocdepth}{2}
\tableofcontents{}

\clearpage

\todo[inline]{upload GISC report to NIFA}

\section{Preface}

I was hired by the University of Guam on October 1, 2003 under a limited-term,
split appointment (50\% extension and 50\% research). On June 26,
2008, I started a tenure-track appointment as extension entomologist
(100\% extension) with the academic rank of assistant professor. At
the end of the 2012 fall term I applied for tenure and promotion to associate professor and
received both in 2013. At the end of 2018 fall term I applied for promotion to
full professor and was promoted on July 11, 2019. 

I work within the Agriculture and Natural Resources Unit of the University
of Guam Cooperative Extension Service. I am a member of the
Environmental Science Graduate Program Faculty and a member of the Western
Pacific Tropical Research Center. 

I anticipate retirement from the University on October 1, 2023.

\vspace{0.5in}
\textbf{Note to Reader:}
If you are reading the PDF version of this report on a device connected
to the internet, you will be able to follow hypertext links to documents
I have referenced.

\href{https://aubreymoore.github.io/CFES2020-22/CFES2023.pdf}{Click here to download the latest version of this document.}


\begin{comment}
I was hired by the University of Guam on October 1, 2003 under a limited-term,
split appointment (50\% extension and 50\% research). On June 26,
2008, I started a tenure-track appointment as extension entomologist
(100\% extension) with the academic rank of assistant professor. At
the end of the 2012 fall term I applied for tenure and promotion to associate professor and
received both in 2013. At the end of 2018 fall term I applied for promotion to
full professor and was promoted on July 11, 2019. 

I work within the Agriculture and Natural Resources Unit of the University
of Guam Cooperative Extension Service. I am a faculty member of the
Environmental Science Graduate Program and a member of the Western
Pacific Tropical Research Center. 

\textbf{Note to Reader:}

This most recent version of this report is available as a PDF format which can be downloaded from \\
\url{to-be-added}. 

If you are reading the PDF version of this report on a device connected
to the internet, you will be able to follow hypertext links to documents
I have referenced.
\end{comment}

\pagebreak

\section{Part I. Role Assignments}
\paragraph{Time and Effort for Each Role}
\begin{itemize}
	\item 51\% Extension and Community Activities (ECA)
	\item 34\% Creative/Scholarly Activity or Research (CSR)
	\item 15\% University and Community Service (UCS)
\end{itemize}

\section{Part II. Extension and Community Activities (ECA) 51\%}

\subsection{NIFA Classification}

\textbf{Critical Issues (CI)}
\begin{itemize}
\item Sustained Agricultural Production Systems
\item Plant Pest Efforts \& Protect Resources of Guam
\end{itemize}

\textbf{NIFA Science Emphasis Areas (SEA)}
\begin{itemize}
\item \textbf{Sustainable Agricultural Production Systems} Addresses human interaction between science, technology, and agriculture. Integrates the biological, physical, and environmental and socioeconomic factors essential to successful production enterprises and viable rural communities. 
\end{itemize}

\textbf{NIFA Knowledge Areas (KA)}
\begin{itemize}
\item KA 203 – Plant Biological Efficiency and Abiotic Stresses Affecting
\item KA 211 – Insects, Mites and Other Anthropods Affecting Plants
\item KA 215 – Biological Control of Pests Affecting Plants
\item KA 216 – Integrated Pest Management Systems
\end{itemize}

\textbf{Planned Activities (PA)}
\begin{itemize}
\item PA1 Insect Diagnostic Services
\item PA2 Detection and Documentation of Invasive Species
\item PA3 University of Guam Insect Collection
\item PA4 Mitigation of Damage to Guam's Ecosystems by Invasive Species
\item PA5 National Plant Diagnostic Network (NPDN)
\item PA6 Guam Invasive Species Advisory Committee (GISAC) and Guam Invasive Species Council (GISC)
\item PA7 Public Outreach: Internet
\item PA8 Public Outreach: Presentations
\item PA9 Public Outreach: Miscellaneous
\item PA10 Public Outreach: Public GitHub Repositories
\end{itemize}

\subsection{ECA-PA1 Insect Diagnostic Services}
\begin{refsection}

\paragraph{Description}	
As an extension entomologist, a major part of my job is providing
insect identification and pest control recommendations to a diverse
clientele including commercial growers, gardeners, householders, GovGuam
agencies, federal agencies, and UOG colleagues. Most client contacts
are initiated by a phone call or a visit by the client to the ANR
office. In many cases identification and pest control recommendations
require a site visit by me and/or extension associates to collect
samples and define the problem.

\paragraph{Activities}
The number of extension calls requiring my assistance averaged approximately
one per day during the reporting period. Many of these are documented
as postings to iNaturalist \cite{moore_inat_from_2020-06-26_to_2023-06-26}.

\begin{comment}
\paragraph{Plans for 15JUN2021-14JUN2021}
I plan to continue providing insect diagnostic services.

\paragraph{Plans for 15JUN2022-14JUN2023}
I plan to continue providing insect diagnostic services.
\end{comment}
\paragraph{References}

\printbibliography[heading=none]
\end{refsection}

\subsection{ECA-PA2 Detection and Documentation of Invasive Species}
\begin{refsection}

\paragraph{Description}
Invasive insects are arriving on Guam at a very high rate (estimates
range as high as one new species per day). Very few of these are detected
and even fewer are identified because Guam suffers from \href{https://en.wikipedia.org/wiki/Taxonomic_impediment}{the taxonomic impediment}.
Even when reliable species determinations are made, new island records
are only rarely documented in the scientific press. Thus, impacts
of invasive insects on Guam and elsewhere in Micronesia are grossly
underestimated. One of my professional goals is to work towards solving
this problem by increasing the detection rate, getting specimens identified
by qualified taxonomists, and publishing new island records in the
scientific literature.

\paragraph{Activities}
iNaturalist was used to document new records for insects detected in Guam and other Micronesian Islands \cite{inatSearchForNew}. 
Three new island records for insects on Guam were documented in iNaturalist posts during the reporting period \cite{inatAleurocanthus_woglumi, inatEncarsia_noyesi, inatConotelus_mexicanus}.

\begin{comment}
\paragraph{Plans for 15JUN2021-14JUN2022}

I will continue to document new island records of insects detected in Micronesia.

The International Union for Conservation of Nature (IUCN-ISSG) is
building a Global Register of Introduced and Invasive Species. I have
volunteered to coordinate building a check list for species on Guam.

The Guam Invasive Species Council is required to maintain a list on
invasive species on Guam. I have volunteered to be ``registrar''
for this list.

\paragraph{Plans for 15JUN2022-14JUN2023}

I will continue to document new island records of insects detected in Micronesia.

The International Union for Conservation of Nature (IUCN-ISSG) is
building a Global Register of Introduced and Invasive Species. I have
volunteered to coordinate building a check list for species on Guam.

The Guam Invasive Species Council is required to maintain a list on
invasive species on Guam. I have volunteered to be ``registrar''
for this list.
\end{comment}
\paragraph{References}
\printbibliography[heading=none]
\end{refsection}


\subsection{ECA-PA3 University of Guam Insect Collection}
\begin{refsection}
	
\paragraph{Description}

The UOG insect collection is a valuable reference collection for extension
entomology, teaching and research. I am a member of the board of directors
for the collection and I work with Dr. Ross Miller to curate and catalog
this collection.

In 2018 I ported the digital catalog for the UOG Insect Collection from a
CSIRO BioLink database to a more modern web-based Symbiota database
which is publicly available online \cite{moore_scan_2023}. I also established an internship to train entomology students how to curate an institutional insect collection and how to add specimen images to the digital catalog\cite{moore_internship_2018}. However, this work came to a halt because of space limitations. 

Facilities provided for the UOG insect collection are very poor. The collection is literally \textit{moth balled} in a small storage room which is too small for essential equipment such as microscopes and cameras. Curation and digitization necessitates removing specimens from the collection and transporting them outdoors to a lab where there is working space and equipment.

\paragraph{Activities}

In 2022, I arranged for EPSCOR funding to have a door installed between the UOG insect collection and the adjacent ANR lab. Access to bench space in the lab will partially solve the space limitation problem described in the preceding section. During 2023, moth balls were added to all specimen drawers. As a result of Typhoon Mwar, the collection was without air conditioning for about three weeks.

\paragraph{References}
\printbibliography[heading=none]
\end{refsection}

\subsection{ECA-PA4 Mitigation of Damage to Guam's Ecosystems by Invasive Species}
\begin{refsection}

Guam's ecosystems are rapidly being degraded by invasive species. These include:
\begin{itemize}
	\item \textbf{Brown treesnake} which has extirpated Guam's forest birds, causing loss of ecosystem services they provided, such as seed dispersal, insect control and pollination.
	\item \textbf{Cycad aulacaspis scale insect}, ACS, which has killed more than 90\% of Guam's endemic cycads, known locally as \textit{fadang}. Fadang went from being the most abundant plant in Guam's forests in 2002 to being listed as an endangered species in 2015.
	\item \textbf{Coconut rhinoceros beetle}, CRB, which is killing coconut palms and palma brava throughout the island. These two palm species where the second and third most abundant trees in Guam's forest in 2002. 
\end{itemize}

Clearly, ecological disasters are happening on Guam, especially in forest ecosystems. As an extension entomologist, I am tasked with providing solutions to problems caused by insect pests. Unfortunately, there are no known methods for effectively controlling CAS and CRB on Guam. Therefor, I spend much of my time and effort performing applied research in an attempt to adequately control CAS and CRB so that restoration of Guam's forests can be attempted. 

\paragraph{Activities}

A new program entitled \textit{Plant Protection and Biosecurity for Guam and Micronesia} was submitted to the National Institute of Food and Agriculture. This program has been approved but not yet funded \cite{NIFA2022}.

Applied research is reported  in the Creative/Research/Scholarly in a section for CRB and a section for CAS.

\paragraph{References}
\printbibliography[heading=none]
\end{refsection}


\subsection{ECA-PA5 National Plant Diagnostic Network (NPDN)}
\begin{refsection}

\paragraph{Description}	

I served as the UOG Coordinator for the National Plant Diagnostic Network (NPDN). UOG receives about \$15K per year from NPDN as a subrecipient of the Western Plant Diagnostic Network administered by UC Davis. 

%Grant details are in  sections \ref{WPDN2}, \ref{WPDN3} and \ref{WPDN4YR}.

\paragraph{Activities}

\begin{itemize}
	\item Participated in monthly conference calls.
	\item Prepared and submitted annual reports.
	\item Prepared a \href{https://github.com/aubreymoore/WPDN/raw/main/4year/WPDN\%20FY23-FY26\%20workplan\%20and\%20budget\%20for\%20UOG.pdf}{four-year work plan and budget for FY2023-FY2026}.
	\item Made a presentation entitled \textit{The Invasive Species Problem on Guam} \cite{moore_invasive_2022} at the 2022 WPDN annual meeting (via Zoom).
	\item In June 2023, I resigned as WPDN coordinator.
	I am currently working with the new WPDN coordinator, Jesse Bamba, to ensure a smooth transition.
\end{itemize}

\paragraph{References}
\printbibliography[heading=none]
\end{refsection}	

\subsection{ECA-PA6 Guam Invasive Species Advisory Committee (GISAC) and Guam Invasive Species Council (GISC)}
\begin{refsection}
	
I am a founding member and regular participant in GISAC. President Underwood delegated me to represent UOG as a voting member of GISC and President Krise reconfirmed my delegation.

\paragraph{Activities}

I participate in GISAC and GISC meetings. Unfortunately, there have been none during the current reporting period.

A GISC meeting was scheduled at Adelup on May 24, 2023. However, this was the day that Typhoon Mwar visited Guam. This GISC meeting has been postponed with a date yet to be set. I prepared a report on invasive species activities at the University of Guam \cite{GISC2023}.

\paragraph{References}
\printbibliography[heading=none]
\end{refsection}	

\subsection{ECA-PA7 Public Outreach: Internet}

Since the 1990s, I have built and maintained web sites to facilitate sharing information about insects in Micronesia. I created a wiki site to serve as an index to web resources I have developed (Available at  \url{https://guaminsects.net/aubwiki2020}). I will continue to use web sites to facilitate sharing information on Guam's insects.
\begin{comment}
\paragraph{Plans for 15JUN2021-14JUN2022}
I will continue to create and maintain websites about insects in Micronesia.

\paragraph{Plans for 15JUN2022-14JUN2023}
I will continue to create and maintain websites about insects in Micronesia.
\end{comment}

\newpage
\subsection{ECA-PA8 Public Outreach: Presentations}
\begin{refsection}

I helped to organize a meeting of the Guam Native Plant Society which was held at the UOG ALS Teaching Lab (ALS124) on Saturday, January 21, and I gave a presentation at this meeting (REF). The topic of this meeting was \textit{Guam insects associated with \textit{Glochidian}} \cite{mooreGlochidion2023}.

On Saturday, April 15, 2023, I made a presentation as part of a workshop on growing hot peppers on Guam \cite{mooreInterestingFactsChili2023}.

\paragraph{References}
\printbibliography[heading=none]
\end{refsection}

\subsection{ECA-PA9 Public Outreach: Miscelleaneous}

On March 30 2023 I participated in a 4H workshop for military children by making presentations on local insects.

\subsection{ECA-PA10 Public Outreach: Public GitHub Repositories}

I attempt to provide access to as much of my work as possible using public GitHub repositories. GitHub is a free service for backing up and sharing documents on the web. Repositories which I have created or updated during the reporting period are listed in Table \ref{repolist}.  The document you
are reading, all previous versions, and related data are stored in a public GitHub directory
at
https://github.com/aubreymoore/CFES2020-22. The document can be read online or
downloaded as a PDF from
https://github.com/aubreymoore/CFES2020-22/blob/main/
CFES2021-06-15to2022-06-14.pdf

I also use GitHub pages for serving static websites. A couple of good example sites are one which I created for my \href{https://aubreymoore.github.io/ALBI-345/}{ALBI345 General Entomology} course and one which is a \href{https://aubreymoore.github.io/crop-pest-list/}{List of Insects and Mites Attacking Crops in Micronesia}.

I will continue to create and maintain public GitHub repositories to provide access to my work as \href{https://en.wikipedia.org/wiki/Open_science}{open science}.

%\begin{longtable}{rl}
\caption{List of GitHub repositories created or pushed between 2022-06-26 and 2023-06-26.This list was generated by \url{code/repo-list-generator.ipynb}.}
\label{repolist}\\
\toprule
 n &                                                                                                       repository \\
\midrule
\endfirsthead
\caption[]{List of GitHub repositories created or pushed between 2022-06-26 and 2023-06-26.This list was generated by \url{code/repo-list-generator.ipynb}.} \\
\toprule
 n &                                                                                                       repository \\
\midrule
\endhead
\midrule
\multicolumn{2}{r}{{Continued on next page}} \\
\midrule
\endfoot

\bottomrule
\endlastfoot
 1 &                           \href{https://github.com/aubreymoore/2020-DOI-CRB-Biocontrol}{2020-DOI-CRB-Biocontrol} \\
 2 &             \href{https://github.com/aubreymoore/2020-FS-CRB-biocontrol-project}{2020-FS-CRB-biocontrol-project} \\
 3 &                                             \href{https://github.com/aubreymoore/albi345-slides}{albi345-slides} \\
 4 &                                               \href{https://github.com/aubreymoore/aleurocanthus}{aleurocanthus} \\
 5 &                                                               \href{https://github.com/aubreymoore/bamba}{bamba} \\
 6 &                                                         \href{https://github.com/aubreymoore/CRB-FIDL}{CRB-FIDL} \\
 7 &                                           \href{https://github.com/aubreymoore/crb-flight-test}{crb-flight-test} \\
 8 &                                                           \href{https://github.com/aubreymoore/crb-mex}{crb-mex} \\
 9 &                                                     \href{https://github.com/aubreymoore/crb-mexico}{crb-mexico} \\
10 &                                       \href{https://github.com/aubreymoore/crb-roadside-code}{crb-roadside-code} \\
11 &                                                 \href{https://github.com/aubreymoore/crb-symbiont}{crb-symbiont} \\
12 &                                                           \href{https://github.com/aubreymoore/crb-vdc}{crb-vdc} \\
13 &                                               \href{https://github.com/aubreymoore/crb-yona-tiff}{crb-yona-tiff} \\
14 &                                                           \href{https://github.com/aubreymoore/crbdist}{crbdist} \\
15 &     \href{https://github.com/aubreymoore/CRBG-action-group-webinar-20201209}{CRBG-action-group-webinar-20201209} \\
16 &                                                             \href{https://github.com/aubreymoore/crbmex}{crbmex} \\
17 &                                                       \href{https://github.com/aubreymoore/crbmexico}{crbmexico} \\
18 &                                             \href{https://github.com/aubreymoore/crop-pest-list}{crop-pest-list} \\
19 &                   \href{https://github.com/aubreymoore/data-mining-insects-of-guam}{data-mining-insects-of-guam} \\
20 &                                           \href{https://github.com/aubreymoore/detection-range}{detection-range} \\
21 &                                                         \href{https://github.com/aubreymoore/GISC2023}{GISC2023} \\
22 &     \href{https://github.com/aubreymoore/globalbioticinteractions.github.io}{globalbioticinteractions.github.io} \\
23 &                                                     \href{https://github.com/aubreymoore/GloBI-test}{GloBI-test} \\
24 &         \href{https://github.com/aubreymoore/Guam-biodiversity-inventory-2023}{Guam-biodiversity-inventory-2023} \\
25 &                 \href{https://github.com/aubreymoore/Guam-CRB-Damage-Map-20021-09}{Guam-CRB-Damage-Map-20021-09} \\
26 &                   \href{https://github.com/aubreymoore/Guam-CRB-Damage-Map-2021-08}{Guam-CRB-Damage-Map-2021-08} \\
27 &                   \href{https://github.com/aubreymoore/Guam-CRB-Damage-Map-2022-02}{Guam-CRB-Damage-Map-2022-02} \\
28 &                   \href{https://github.com/aubreymoore/Guam-CRB-Damage-Map-2022-11}{Guam-CRB-Damage-Map-2022-11} \\
29 &                   \href{https://github.com/aubreymoore/Guam-CRB-Damage-Map-2023-04}{Guam-CRB-Damage-Map-2023-04} \\
30 &                                       \href{https://github.com/aubreymoore/Guam-CRB-web-maps}{Guam-CRB-web-maps} \\
31 &                   \href{https://github.com/aubreymoore/Guam-Forestry-Workshop-2022}{Guam-Forestry-Workshop-2022} \\
32 &                                               \href{https://github.com/aubreymoore/guam-ias-bolo}{guam-ias-bolo} \\
33 & \href{https://github.com/aubreymoore/Guam-insect-associates-of-Glochidion}{Guam-insect-associates-of-Glochidion} \\
34 &                                             \href{https://github.com/aubreymoore/Harmonic-Radar}{Harmonic-Radar} \\
35 &                                                   \href{https://github.com/aubreymoore/interaction}{interaction} \\
36 &                                                                 \href{https://github.com/aubreymoore/IREC}{IREC} \\
37 &                                         \href{https://github.com/aubreymoore/McIntire-Stennis}{McIntire-Stennis} \\
38 &                                                           \href{https://github.com/aubreymoore/mexico1}{mexico1} \\
39 &                                                         \href{https://github.com/aubreymoore/OccuTree}{OccuTree} \\
40 &                               \href{https://github.com/aubreymoore/palau-guts-experiment}{palau-guts-experiment} \\
41 &                                       \href{https://github.com/aubreymoore/PESC-OIA-overview}{PESC-OIA-overview} \\
42 &                                                                 \href{https://github.com/aubreymoore/pika}{pika} \\
43 &                                                                   \href{https://github.com/aubreymoore/POW}{POW} \\
44 &                                                             \href{https://github.com/aubreymoore/py4web}{py4web} \\
45 &                       \href{https://github.com/aubreymoore/rethinking-OrNV-bioassays}{rethinking-OrNV-bioassays} \\
46 &                                   \href{https://github.com/aubreymoore/sticky-trap-imaging}{sticky-trap-imaging} \\
47 &                                             \href{https://github.com/aubreymoore/TinianTrip2022}{TinianTrip2022} \\
48 &                                     \href{https://github.com/aubreymoore/young-tekken-award}{young-tekken-award} \\
\end{longtable}


\begin{longtable}{rl}
	\caption{List of GitHub repositories created or pushed between 2022-06-26 and 2023-06-26.This list was generated by \url{code/repo-list-generator.ipynb}.}
	\label{repolist}\\
	\toprule
	n &                                                                                                       repository \\
	\midrule
	\endfirsthead
	\caption[]{List of GitHub repositories created or pushed between 2022-06-26 and 2023-06-26.This list was generated by \url{code/repo-list-generator.ipynb}.} \\
	\toprule
	n &                                                                                                       repository \\
	\midrule
	\endhead
	\midrule
	\multicolumn{2}{r}{{Continued on next page}} \\
	\midrule
	\endfoot
	
	\bottomrule
	\endlastfoot
	1 &                           \href{https://github.com/aubreymoore/2020-DOI-CRB-Biocontrol}{2020-DOI-CRB-Biocontrol} \\
	2 &             \href{https://github.com/aubreymoore/2020-FS-CRB-biocontrol-project}{2020-FS-CRB-biocontrol-project} \\
	3 &                                             \href{https://github.com/aubreymoore/albi345-slides}{albi345-slides} \\
	4 &                                               \href{https://github.com/aubreymoore/aleurocanthus}{aleurocanthus} \\
	5 &                                                               \href{https://github.com/aubreymoore/bamba}{bamba} \\
	6 &                                                         \href{https://github.com/aubreymoore/CRB-FIDL}{CRB-FIDL} \\
	7 &                                           \href{https://github.com/aubreymoore/crb-flight-test}{crb-flight-test} \\
	8 &                                                           \href{https://github.com/aubreymoore/crb-mex}{crb-mex} \\
	9 &                                                     \href{https://github.com/aubreymoore/crb-mexico}{crb-mexico} \\
	10 &                                       \href{https://github.com/aubreymoore/crb-roadside-code}{crb-roadside-code} \\
	11 &                                                 \href{https://github.com/aubreymoore/crb-symbiont}{crb-symbiont} \\
	12 &                                                           \href{https://github.com/aubreymoore/crb-vdc}{crb-vdc} \\
	13 &                                               \href{https://github.com/aubreymoore/crb-yona-tiff}{crb-yona-tiff} \\
	14 &                                                           \href{https://github.com/aubreymoore/crbdist}{crbdist} \\
	15 &     \href{https://github.com/aubreymoore/CRBG-action-group-webinar-20201209}{CRBG-action-group-webinar-20201209} \\
	16 &                                                             \href{https://github.com/aubreymoore/crbmex}{crbmex} \\
	17 &                                                       \href{https://github.com/aubreymoore/crbmexico}{crbmexico} \\
	18 &                                             \href{https://github.com/aubreymoore/crop-pest-list}{crop-pest-list} \\
	19 &                   \href{https://github.com/aubreymoore/data-mining-insects-of-guam}{data-mining-insects-of-guam} \\
	20 &                                           \href{https://github.com/aubreymoore/detection-range}{detection-range} \\
	21 &                                                         \href{https://github.com/aubreymoore/GISC2023}{GISC2023} \\
	22 &     \href{https://github.com/aubreymoore/globalbioticinteractions.github.io}{globalbioticinteractions.github.io} \\
	23 &                                                     \href{https://github.com/aubreymoore/GloBI-test}{GloBI-test} \\
	24 &         \href{https://github.com/aubreymoore/Guam-biodiversity-inventory-2023}{Guam-biodiversity-inventory-2023} \\
	25 &                 \href{https://github.com/aubreymoore/Guam-CRB-Damage-Map-20021-09}{Guam-CRB-Damage-Map-20021-09} \\
	26 &                   \href{https://github.com/aubreymoore/Guam-CRB-Damage-Map-2021-08}{Guam-CRB-Damage-Map-2021-08} \\
	27 &                   \href{https://github.com/aubreymoore/Guam-CRB-Damage-Map-2022-02}{Guam-CRB-Damage-Map-2022-02} \\
	28 &                   \href{https://github.com/aubreymoore/Guam-CRB-Damage-Map-2022-11}{Guam-CRB-Damage-Map-2022-11} \\
	29 &                   \href{https://github.com/aubreymoore/Guam-CRB-Damage-Map-2023-04}{Guam-CRB-Damage-Map-2023-04} \\
	30 &                                       \href{https://github.com/aubreymoore/Guam-CRB-web-maps}{Guam-CRB-web-maps} \\
	31 &                   \href{https://github.com/aubreymoore/Guam-Forestry-Workshop-2022}{Guam-Forestry-Workshop-2022} \\
	32 &                                               \href{https://github.com/aubreymoore/guam-ias-bolo}{guam-ias-bolo} \\
	33 & \href{https://github.com/aubreymoore/Guam-insect-associates-of-Glochidion}{Guam-insect-associates-of-Glochidion} \\
	34 &                                             \href{https://github.com/aubreymoore/Harmonic-Radar}{Harmonic-Radar} \\
	35 &                                                   \href{https://github.com/aubreymoore/interaction}{interaction} \\
	36 &                                                                 \href{https://github.com/aubreymoore/IREC}{IREC} \\
	37 &                                         \href{https://github.com/aubreymoore/McIntire-Stennis}{McIntire-Stennis} \\
	38 &                                                           \href{https://github.com/aubreymoore/mexico1}{mexico1} \\
	39 &                                                         \href{https://github.com/aubreymoore/OccuTree}{OccuTree} \\
	40 &                               \href{https://github.com/aubreymoore/palau-guts-experiment}{palau-guts-experiment} \\
	41 &                                       \href{https://github.com/aubreymoore/PESC-OIA-overview}{PESC-OIA-overview} \\
	42 &                                                                 \href{https://github.com/aubreymoore/pika}{pika} \\
	43 &                                                                   \href{https://github.com/aubreymoore/POW}{POW} \\
	44 &                                                             \href{https://github.com/aubreymoore/py4web}{py4web} \\
	45 &                       \href{https://github.com/aubreymoore/rethinking-OrNV-bioassays}{rethinking-OrNV-bioassays} \\
	46 &                                   \href{https://github.com/aubreymoore/sticky-trap-imaging}{sticky-trap-imaging} \\
	47 &                                             \href{https://github.com/aubreymoore/TinianTrip2022}{TinianTrip2022} \\
	48 &                                     \href{https://github.com/aubreymoore/young-tekken-award}{young-tekken-award} \\
\end{longtable}

\newpage
\section{Part II. Creative/Scholarly Activities or Research (CSR) 34\%}

\subsection{NIFA Classification}

\textbf{Critical Issues (CI)}
\begin{itemize}
	\item Sustained Agricultural Production Systems
	\item Plant Pest Efforts \& Protect Resources of Guam
\end{itemize}

\textbf{NIFA Science Emphasis Areas (SEA)}
\begin{itemize}
	\item \textbf{Sustainable Agricultural Production Systems} Addresses human interaction between science, technology, and agriculture. Integrates the biological, physical, and environmental and socioeconomic factors essential to successful production enterprises and viable rural communities. 
\end{itemize}

\textbf{NIFA Knowledge Areas (KA)}
\begin{itemize}
	\item KA 203 – Plant Biological Efficiency and Abiotic Stresses Affecting
	\item KA 211 – Insects, Mites and Other Anthropods Affecting Plants
	\item KA 215 – Biological Control of Pests Affecting Plants
	\item KA 216 – Integrated Pest Management Systems
	\item KA 901 - Project, and Program Design, and Statistics
	\item KA 902 - Administration of Projects and Programs
\end{itemize}

\textbf{Planned Activities (PA)}
\begin{itemize}
	\item PA1 Peer Reviewed Publications (N=4)
	\item PA2 Publications Submitted for Peer Review (N=1)
	\item PA3 Journal Articles in Preparation (N=4)
	\item PA4 Coconut Rhinoceros Beetle (CRB) Biocontrol
	\item PA5 Guam Biodiversity Inventory
	\item PA6 Cycad Aulacaspis Scale (CAS) Biocontrol
	\item PA7 Eight Spot Butterfly (ESB) Conservation
\end{itemize}


\subsection{CSR-PA1 Peer Reviewed Publications (N=4)}
	\begin{refsection}
		\nocite{cave_biological_2022, mooreProposalDetectingCoconut2022,caasiMathematicalModelInvasion2023, mooreCoconutRhinocerosBeetle2023}
		\printbibliography[heading=none]	
	\end{refsection}

\subsection{CSR-PA2 Publication submitted for Peer Review (N=1)}
\begin{refsection}
	\nocite{paudelPheromone2023}
	\printbibliography[heading=none]	
\end{refsection}

\subsection{CSR-PA3 Journal Articles in Preparation (n=5)}
\begin{refsection}
	\nocite{moore_first_nodate-1,moore_mariana_2013,moore_three_nodate-1,moore_change_nodate,moore_coconut_nodate-1}	
	\printbibliography[heading=none]		
\end{refsection}

\subsection{CSR-PA4 Coconut Rhinoceros Beetle (CRB) Biocontrol and Related Activity}\label{sec:coconut-rhinoceros-beetle-(crb)-biocontrol}
\begin{refsection}
	
\paragraph{Description}

A newly discovered biotype of coconut rhinoceros beetle (CRB-G) is
rapidly killing coconuts and other palms on Guam and on other Pacific
islands. Following a failed eradication attempt on Guam, CRB-G proved
hard to control because it is resistant to \emph{Oryctes rhinoceros}
nudivirus (OrNV), which was previously used as the preferred biological
control agent for control of CRB outbreaks on Pacific Islands and
elsewhere. Prior to the discovery of CRB-G, all OrNV releases on
Pacific Islands resulted in immediate and sustained suppression of
CRB damage to low levels and prevented tree mortality.

Guam is currently experiencing an uncontrolled and unmonitored island-wide
CRB-G outbreak which was triggered by abundant CRB-G breeding sites
in the form of dead and dying vegetation left in the wake of Typhoon
Dolphin which occurred in May 2015. Most of these
breeding sites are inaccessible to sanitation efforts, being either deep
in the jungle or on military land (which covers one third of Guam).
A positive feedback cycle has begun whereby large numbers of adult
beetles are killing large numbers of palms which become breeding sites
which generate even higher numbers of adults. Severe damage to Guam's
palms prompted the Governor of Guam to declared a state of emergency
in July 2017.

The main objective of this project is to stop the uncontrolled outbreak
on Guam. Entomologists working on the CRB-G problem on several Pacific
islands agree that the most feasible tactic to halt tree mortality
and suppress damage to tolerable levels is establishment of biological
control using an isolate of OrNV which is highly effective as a biological
control agent for CRB-G. We are working with collaborators to identify
populations of CRB-G throughout the Asia-Pacific region. We will sample
these populations for biological control agent candidates which will
be evaluated in laboratory bioassays performed at UOG. Promising candidates
will be field released using autodissemmination as per a USDA-APHIS
import and release permit.

Concurrent with establishment of CRB-G biocontrol, success of the
project will be monitored in a quarterly, island-wide tree health
survey and incidence of OrNV infection will be monitored in a subsample
of all field collected CRB-G.

If the Guam CRB-G infestation cannot be controlled, it is expected
that most palms on the island will be killed and CRB-G will continue
to spread to other islands and beyond. If CRB-G invades smaller islands
and atolls where coconut is the tree of life, a human tragedy will
ensue. On larger islands, coconut and oil palm industries will be
severely impacted. Attempts to organize a well-funded, coordinated regional project in response to CRB-G have failed underway. However, UOG plays a major role in the \textit{ad hoc} CRB-G Action Group which was established to facilitate sharing scientific/technical information among people working on the CRB-G problem.

\paragraph{Funding and Project Management} This is my largest and most important project, requiring a lot of time and effort for project management including preparation of grant proposals and reports. During the reporting period, funding was provided by 2 grants totaling \$361,234: OIA-CRB, FS-CRB and FS-CRB-HR.  Details, including links to project proposals, work plans and progress reports are available in the grants section of this report.

\paragraph{Staffing}

Grant funding for this project supports 2 full-time technicians, Christian Cayanan and Laura Caser, and two part-time technicians, Leah Bukurou and Ed Bayson.

\paragraph{CRB-G biocontrol} 

We will continue performing bioassays until a potential OrNV biocontrol candidate is found. Once we have one, we will begin propagation \textit{in vivo} and field releases via autodissemination. I already have a USDA-APHIS permit for field release of OrNV.

The current priority is to perform a critical laboratory bioassay to test for significant reduction in fecundity (egg laying) caused by OrNV isolate V23B. This isolate caused significant mortality in bioassays performed in our lab and also in the Solomon Islands. However, according to some of the literature, the main mode of action causing population reduction by OrNV is not mortality but reduction in fecundity. Given the long lifespan of CRB, this experiment will require at least a year (includes lab rearing of test insects).

\paragraph{Establishment of a CRB Rearing Facility and Rearing Protocol}

Development of biocontrol for CRB-G will require laboratory bioassays using standardized, healthy lab-reared beetles of equivalent age. Previously, we used beetles collected from pheromone traps for this purpose. However, mortality in experimental control groups was highly variable, yielding irreproducible results. 

During the reporting period:
\begin{itemize}
	\item We built and equipped a CRB rearing facility in a 40 foot shipping container.  
	\item We developed and tested a natural larval diet by grinding dead standing coconut stems containing CRB breeding sites.
	\item A recent experiment shows that the growth rate of CRB larvae is greatly increased by adding field-collected larval frass to neonate larvae. We suspect that this increase in growth rate is due to addition of gut symbionts present in the frass.	
\end{itemize}

\paragraph{Establishment of an Island-wide CRB Damage Monitoring System}

We developed an island-wide roadside monitoring system to track spatial and temporal changes in CRB damage levels. Data are collected using a smart phone attached to a project vehicle. The phone continually records georeferenced roadside images as the vehicle is driven along all major results in both directions. Data are automatically analysed by an image analysis system which detects coconut palms in the images, calculates a damage index for each palm, and outputs results on a map of Guam. A nontechnical overview of this system was published in the 2020 WPTRC Impact Report \parencite{moore_using_2020-1}.

To date, we have completed eight island-wide surveys and data for a ninth survey were recorded in June 2023 but have not yet been analyzed. Interactive damage maps and data are available on the web: 

\begin{itemize}
	
\item Survey completed 2020-10-22: 
\href{(https://aubreymoore.github.io/new-crb-damage-map}{View map} | \href{https://github.com/aubreymoore/new-crb-damage-map}{View GitHub repository} | \href{https://github.com/aubreymoore/new-crb-damage-map/raw/main/original-map/Guam01.db}{Download database}

\item Survey completed 2020-12-24:
\href{https://aubreymoore.github.io/Guam-CRB-damage-map-2020-12/webmap/v1}{View map} | 
\href{https://github.com/aubreymoore/Guam-CRB-damage-map-2020-12)}{View GitHub repository} | 
\href{https://github.com/aubreymoore/Guam-CRB-damage-map-2020-12/raw/main/Guam02.db}{Download database}

\item Survey completed 2021-03-18: 
\href{https://aubreymoore.github.io/Guam-CRB-Damage-Map-2021-03}{View map} | 
\href{https://github.com/aubreymoore/Guam-CRB-damage-map-2021-03}{View GitHub repository} | 
\href{https://github.com/aubreymoore/Guam-CRB-Damage-Map-2021-03/blob/main/Guam03.zip}{Download database}

\item Survey completed 2021-05-20: 
\href{https://aubreymoore.github.io/Guam-CRB-Damage-Map-2021-05/webmap}{View map} | 
\href{https://github.com/aubreymoore/Guam-CRB-Damage-Map-2021-05}{View GitHub repository} | 
\href{https://github.com/aubreymoore/Guam-CRB-Damage-Map-2021-05/raw/main/Guam04.db}{Download database}

\item Survey completed 2021-09-03: 
\href{https://aubreymoore.github.io/Guam-CRB-Damage-Map-20021-09/webmap}{View map} | 
\href{https://github.com/aubreymoore/Guam-CRB-Damage-Map-20021-09}{View GitHub repository} | 
\href{https://github.com/aubreymoore/Guam-CRB-Damage-Map-20021-09/raw/main/crb-damage-2021-09.db}{Download database}

\item Survey completed 2022-02-23: 
\href{https://aubreymoore.github.io/Guam-CRB-Damage-Map-2022-02/webmap/#11/13.4483/144.7860}{View map} | 
\href{https://github.com/aubreymoore/Guam-CRB-Damage-Map-2022-02}{View GitHub repository} | 
\href{https://github.com/aubreymoore/Guam-CRB-Damage-Map-2022-02/raw/main/Guam-CRB-Damage-Map-2022-02.db}{Download database}

\item Survey completed 2022-11-28: 
\href{https://aubreymoore.github.io/Guam-CRB-Damage-Map-2022-11/webmap/#11/13.4483/144.7860}{View map} | 
\href{https://github.com/aubreymoore/Guam-CRB-Damage-Map-2022-11}{View GitHub repository} | 
\href{https://github.com/aubreymoore/Guam-CRB-Damage-Map-2022-11/raw/main/output/Guam07.db}{Download database}

\item Survey completed 2023-04-07:  
\href{https://aubreymoore.github.io/Guam-CRB-Damage-Map-2023-04/webmap/#11/13.4483/144.7860}{View map} | 
\href{https://github.com/aubreymoore/Guam-CRB-Damage-Map-2023-04}{View GitHub repository} | 
\href{https://github.com/aubreymoore/Guam-CRB-Damage-Map-2023-04/raw/main/output/results.db}{Download database}

\end{itemize}

Results of surveys are made available to the Guam public via press releases \cite{moorePressRelease2023}. Survey results show that one in five (20\%) of Guam's coconut palms growing along roadsides are damaged by CRB (Fig. \ref{fig:timeline}).

% TODO: \usepackage{graphicx} required
\begin{figure}[H]
	\centering
	\includegraphics[width=1\linewidth]{images/timeline}
	\caption{Percent of coconut palms growing along Guam's roadsides which show visible damage from coconut rhinoceros beetle.}
	\label{fig:timeline}
\end{figure}



%\paragraph{Current focus} is on finding an isolate of \textit{Oryctes} rhinoceros nudivirus which can be used as a biological control agent for CRB-G. Laboratory bioassays have identified one OrNV isolate which is potential candidate and further tests are under way.
%
%I have developed an online database to facilitate record keeping and report generation for CRB rearing and bioassays \cite{moore_coconut_2019-1}. 
%
%Dr. Grasela has worked in coordination with Dr. Hui Jiang to build DNA diagnostics capacity. We can now test for OrNV in individual beetles.

\paragraph{International collaboration} will be essential for finding a way to halt massive ecological and economic damage to Pacific islands invaded by CRB-G. A \textbf{CRB-G Action Group} was formed was formed to facilitate collaboration and cooperation. prior to COVID, this group met annually at international scientific meetings. During COVID, I helped to keep the group together by hosting Zoom webinars with assistance from the UOG Office of Information Technology. I created web pages to facilitate access recordings of these webinars: 
\begin{itemize}
	\item December 9, 2020 \cite{moore_video_2020}
	\item March 17, 2021 \cite{moore_video_2021-1}
	\item November 23, 2021 \cite{moore_video_2021}	
\end{itemize}

\section{}

The first \textbf{Pacific Ecological Security Conference} (PESC) held in Palau during October 2022, brought together island leaders, development partners, regional organizations, agricultural/food security and natural resource managers, and the media to discuss the importance of managing and preventing the spread of invasive species in Pacific Island environments \cite{anonymousFirstPacificEcological2022}.

Conference hosts and sponsors included Government of Palau Ministry of Agriculture, Fisheries, and Environment, Secretariat of the Pacific Community (SPC), the East-West Center, the Global Environment Facility (GEF), the Nature Conservancy, the US Department of the Interior - Office of Insular Affairs, the U.S. Forest Service, and the Ocean Policy Research Institute.

Delegates from Guam were Roland Quitugua and Aubrey Moore from UOG and Glenn Dulla and Andrea Blas from the Department of Agriculture. The Guam delegation participated in a satellite meeting organized by the U.S. Department of the Interior - Office of Insular Affairs and the U.S. Forest Service. At this meeting each island group presented a status report on invasive species issue (See \parencite{mooreOverviewInvasiveSpecies2022-10-06} for slides from the Guam presentation). This was followed by discussions on federal funding and collaboration.  

\textbf{Strategic Action Plan for Coconut Rhinoceros Beetle (CRB).}
Prior to the PESC, members of the Guam delegation worked with colleagues throughout the Pacific to write a Pacific-wide strategic action plan for response to the recent spread of CRB. This plan was accepted by PESC delegates and has been published online \cite{conferenceStrategicActionPlan2022}.

The plan includes five objectives:

\begin{enumerate}
	\item Enhance Regional Coordination and support of PICTs to achieve the
	objectives of the CRB Strategic Action Plan (SAP)
	\item Conduct immediate and long-term collaborative research to develop
	tools and understanding necessary to enable effective CRB prevention, control, and
	eradication
	\item Prevent the spread of CRB to new locations in the Pacific region and
	beyond
	\item Implement an active early detection, and rapid response system for new
	outbreaks at regional and island levels
	\item Improve implementation of control efforts and mitigate the impacts of
	CRB where already present in the region
	
\end{enumerate} 

\paragraph{Outreach} In an effort to facilitate sharing technical and scientific information among people working on CRB, I have developed and maintain several online resources including a wiki \parencite{moore_crb-g_2019}, a Facebook site \parencite{moore_facebook_2019}, an online interactive map of CRB invasion history \parencite{moore_online_2019} and a CRB reference library \parencite{moore_online_2021}, and an online email discussion site \parencite{moore_online_2021-1}.

To share information with the general public on Guam and elsewhere in Micronesia I initiate and participate in writing press releases and I also interact with reporters. Examples of recent stories published by the local media include:

\begin{itemize}
	
\item \textbf{Entomologist reveals origins of rhino beetle in Marianas}, by Thomas Manglona, KUAM 2023-03-03 \cite{manglonaEntomologistRevealsOrigins2023}

\item \textbf{UOG, Tokyo University test a rhino beetle virus from Palau on Guam’s beetles}, press release by Jackie Hanson, CNAS-UOG 2023-05-04 \cite{hansonUOGTokyoUniversity2023}

\item \textbf{UOG, partners studying virus to curb invasive beetle population} by Richelle Aguon, Guam Daily Post 2023-05-29 \cite{postUOGPartnersStudying2023}

\end{itemize}

\paragraph{Harmonic radar}
%\begin{refsection}
I am investigating the feasibility of using harmonic radar using funding from a US Forest Service grant [\ref{FS-CRB-HR}]. Details of this applied research are available as a preprint journal article \cite{moore_detecting_2022}. This work is being done in collaboration with Dr. Matt Siderhurst, a chemical ecologist at Eastern Mennonite University, Virginia. I am also collaborating with Dr. Glenn Dulla, Guam Department of Agriculture on the feasibility of using a harmonic radar unit attached to a drone. 

SPEND SOME TIME ON THIS
			
%%\printbibliography[heading=none]
%%\end{refsection}
%		
%\paragraph{CRB-G tissue culture}
%%\begin{refsection}
%The project's insect pathologist has established a tissue culture of CRB-G cells \cite{grasela_preliminary_2022}.
%This cell line may be useful for in vitro production of \textit{Oryctes rhinoceros} nudivirus which is currently propagated in a cell line from a different scarab beetle.
%%\printbibliography[heading=none]
%%\end{refsection}
%
%\paragraph{Funding} 
%
%Three of the four grants supporting this project, terminate within calendar year 2022. The remaining grant, DOI-OIA [\ref{OIA-CRB}], terminates on September 30, 2023. I do not have plans to apply for additional grant funding for this project.
%
%\paragraph{Staffing}
%
%The project's insect pathologist, Dr. Jim Grasela, is resigning on June 24, 2022. Recruitment of additional scientific/technical help will be required.


\paragraph{References}
\printbibliography[heading=none]

\end{refsection}


\subsection{CSR-PA5 Guam Biodiversity Inventory}
\begin{refsection}
	
	\paragraph{Description}
	
	I consider this to be my second most important project.
	
	A biodiversity inventory is essentially a database containing a comprehensive
	check list of all taxa known occur within a defined area.
	
	A terrestrial biodiversity inventory for Guam is needed to document
	rapid changes to Guam's ecosystems, to provide free
	and open access to information on Guam's flora and
	fauna, and to share Guam biodiversity information with the global
	scientific community, policy makers and the public.
	
	The Guam Biodiversity Inventory will facilitate automatic generation
	and updates to lists such as: a list of all invasive species on Guam
	with year first recorded, a list of new species described from specimens
	collected on Guam, a list of observations for Guam's
	endangered species, a list of Guam's native plants
	with associated herbivores and pathogens, and a list of crops grown
	on Guam and pests and pathogens which attack them.
	
	\paragraph{Activities}
	
	\paragraph{Funding} This project is supported by my McIntire-Stennis grant \ref{BIODIVERSITY} which terminates on 2022-09-30.
	
	\paragraph{Staffing} I offered an internship to work on this project to Annette Kang, a graduate student from Guam currently working on a PhD in Entomology at Cornell University. Progress on this project was impeded because of a nine month delay in processing my interns stipend payment.  	
	
	\paragraph{Data mining} During the reporting period, the focus within this project was to extract data from legacy entomological literature for Guam, namely \textit{Insects of Guam} I and II. This work was facilitated using a sophisticated workflow developed by \href{https://en.wikipedia.org/wiki/Plazi}{Plazi}, a Swiss-based international non-profit association supporting and promoting the development of persistent and openly accessible digital biodiversity information. The Plazi office in Brazil kindly supported this project by provided online training sessions for myself and Annette. Data extracted from the literature are automatically published on the Global Biodiversity Information Facility as datasets and occurrence records. I wrote an \href{https://aubreymoore.github.io/data-mining-insects-of-guam/validator2/status_report.html}{online dashboard to track our data-mining progress}. More details are available in my 2021 annual report for this project \cite{moore_guam_2021}
\begin{comment}	
\paragraph{Plans for 15JUN2021-14JUN2022}
I am working to complete this project before the grant expires on 2022-09-30 and to complete a final report due 2022-12-31.

\paragraph{Plans for 15JUN2022-14JUN2023}
I am working to complete this project before the grant expires on 2022-09-30 and to complete a final report due 2022-12-31.
\end{comment}
	\paragraph{References}
	\printbibliography[heading=none]
\end{refsection}

%\subsection{Guam Forest Insect Survey}
%\begin{refsection}
%	
%	\paragraph{Description}
%	
%	The objective of this project is to compile a comprehensive check
%	list of insects impacting Guam's forests. While it is notable that
%	Guam's two most numerous forest trees, namely fadang, \emph{Cycas
%		micronesica}, and coconut palm, \emph{Cocos nucifera}, are under simultaneous
%	attack by invasive insects, there are many other forest plants under
%	attack from invasive insects. This project is funded by McIntire-Stennis.
%	
%	\paragraph{Activities}
%	
%	This grant was completed in 2018. See final report \cite{moore_aubreymoore/mcintire-stennis_2018}.
%		
%	\paragraph{Plans}
%	
%	None. This grant project has been completed.
%	
%	\paragraph{References}
%	\printbibliography[heading=none]
%\end{refsection}


\subsection{CSR-PA6 Cycad Aulacaspis Scale (CAS) Biocontrol}
\label{CASbiocontrol}
\begin{refsection}

\paragraph{Description}

A US Forest Service survey published in 2002 reported that the most
abundant tree in Guam's forests (DBH > 5 inches) was Guam's endemic
cycad, \emph{Cycas micronesica}. In 2003, an invasive scale insect,
\emph{Aulacaspis yasumatsui}, was detected on ornamental cycads but
it soon infested wild cycads and started killing them. Within a decade,
90\% of Guam\textquoteright s endemic cycads have been killed by the
scale and other invasive species. \emph{Cycas micronesica} was placed
on the US National Endangered Species List in 2015.

Mature plants are protected by a lady beetle I introduced, but no
natural reproduction of the cycads is occurring because seeds and seedlings are
still being killed by the scale insect. A likely solution to this
problem is establishment of a small biocontrol agent, such as a miniature
parasitic wasp, which will control scale insects infesting seeds and
seedlings.

\paragraph{Activities}

I collaborated with Dr. Jim McConnell on insect pests impacting cycads on Tinian and Guam. I set up a pest monitoring of cycads in conservation plots on Tinian using leaf samples \cite{moore_monitoring_2022} and yellow sticky traps \cite{moore_monitoring_2022-1}.

In March 2022, I hosted a visit from Dr. Ron Cave from the University of Florida. Ron is an expert on CAS biocontrol and I have been trying to get him out here on a consulting trip for several years. Some discussions with USFWS earlier this year led to them funding the trip.

During his trip, Ron and I hosted a Zoom seminar on CAS biocontrol. I put recordings of the presentations and discussion online \cite{cave_biological_2022-1}. After the trip, Dr. Cave provided \href{https://github.com/aubreymoore/CAS-biocontrol-seminar/raw/main/Cave-CAS-report-2022.pdf}{comprehensive consulting report complete with recommendations}. 

In addition, I coauthored a book chapter on CAS biocontrol with Ron Cave and Mark Wright, University of Hawaii.

Note that my work on CAS biocontrol is currently unfunded. \cite{cave_biological_2022}.

\begin{comment}
\paragraph{Plans 15JUN2021-14JUN2022}

I am collaborating with the US Fish and Wildlife Service on prospects for CAS biocontrol. On my suggestion USFWS is considering funding a consulting trip to Guam by Dr. Ron Cave, a CAS biocontrol expert from the University of Florida.  

\paragraph{Plans 15JUN2022-14JUN2023}
I will work to help find funding to implement Dr. Cave's recommendations. 
\end{comment}
\paragraph{References}
\printbibliography[heading=none]

\end{refsection}

\subsection{CSR-PA7 Eight Spot Butterfly (ESB) Conservation}
\begin{refsection}

\paragraph{Description}

The Guam Department of Agriculture Division of Aquatic and Wildlife
Resources (GDOA-DAWR) requested assistance with conservation of the
rare Mariana eight-spot butterfly, \emph{Hypolimnas octocula marianensis}. I prepared a grant proposal and a permit application to do this work under a cooperative agreement with the GDOA-DAWR.

The objective of this project is to investigate the feasibility of captive rearing.

\paragraph{Activities}

I have partnered with Dr. Curt (George) Fiedler, Biology Department, and the Center for Island Sustainability to collaborate on this project.

A large field cage (20x20X10 feet) has been built in the UOG Center for Island Sustainability compound in Dean's Circle and a shade house has been stocked with host plants. 

On 2020-05-25, I organized a Zoom conference call with the US Fish and Wildlife Service to discuss conditions of a new permit which will allow UOG and the Guam Department of Agriculture - Division of Aquatic and Wildlife Resources to work on captive breeding of the Mariana eight-spot butterfly.
\begin{comment}
\paragraph{Plans 15JUN2021-14JUN2022}

I will assist in requesting a new permit from USFWS to work with the eight-spot butterfly. Experimental facilities will be constructed at the Center for Island Sustainability compound on the UOG campus.

\paragraph{Plans 15JUN2022-14JUN2023}

Eight-spot butterfly breeding experiments will commence within the next 2 months.
\end{comment}


\end{refsection}


%\subsection{Development of a Camera Trap for Insects}
%\begin{refsection}
%\paragraph{Description}
%
%The objective of this project is to build a camera trap which uses motion detection to automatically capture short videos of active insects.
%
%The initial target application is a surveillance system for insects visiting flowers.
%
%\paragraph{Activities}
%
%Initial attempts at hardware and software development are available on an Open Science Framework site \cite{moore_development_2019} and in a GitHub repository \cite{moore_github_2019-2}.
%
%\paragraph{Plans}
%
%For the first target application of this technology, I am partnering with Dr. Jim McConnel and staff of the Guam Plant Extinction Prevention Project to discover insect pollinators of an endangered endemic plant.
%
%I plan to test the camera trap for monitoring bee hive activity, including detecting arrival of hornets (\textit{Vespa tropica}).
%
%USDA-APHIS herpetologist, Dr. Shane Sears, has asked me to collaborate with him on developing digital image analysis of brown tree snake videos.
%
%\paragraph{References}
%\printbibliography[heading=none]
%\end{refsection}

\pagebreak
\section{Part II. University and Community Service (UCS) 15\%}


\subsection{UCS-PA2 Graduate Instruction}

I am a graduate faculty member of the Environmental Science program and I serve on master's program committess for 3 students:

\begin{itemize}
	\item Caylin McCormick, a student studying brown tree snake
	\item Ajalyn Omelau, a student studying little fire ant
	\item Matt Putnam, a biology student working on captive breeding of the Marianas eight-spot butterfly
\end{itemize}

\subsection{UCS-PA3 Faculty Committees}

\paragraph{Faculty Building Facilities Committee for the ALS}

This committee was formed by the Agriculture and Life Sciences Division
to consult with the Dean on facilities problems within the Agriculture
and Life Sciences Building. I was re-elected as chair of this committee, joined by Dr. Jim McConnell.

\paragraph{Search Committee: Restoration Ecologist}

I served on a search committee for the new Restoration Ecologist position in the Biology department with Dr. Dan Lidstrom (Chair) and Dr. Frank Camacho.

\paragraph{Search Committee: Research Associates for RCUOG Brown Treesnake Grants}

I was a member of several search committees for BTS technicians. I was joined by Dr. Shane Siers(PI and Chair) and Dr. Dan Lindstrom.

\paragraph{Continuing employment committee}

I served as chair of the continuing employment committee for Jesse Bamba.

\newpage
\section{Grants which were active during the reporting period (n=7; \$494,446)}

During the reporting period, I managed 7 grants totalling \$494,446 (see Table \ref{grantlist} and Fig. \ref{gannt}). Four of these grants expired during the period. Final reports were submitted and accepted. The Western Plant Diagnostic Network grants (WPDN2 and WPDN3) have recently been transferred to Jesse Bamba as the new UOG coordinator for this ongoing project. Two grants funding my CRB work remain active, DOI-OIA and FS-CRB. I will submit final reports for these grants before I retire.

\begin{table}[h!]
	\centering
	\caption{List of grants active during the reporting period (2022-06-26 through 2023-06-26).}
	\label{grantlist}
	\begin{tabular}{lp{3in}>{\raggedleft\arraybackslash}l}
		\toprule
		code &                                                                                                    title &  funding \\
		\midrule
		OIA-CRB (\ref{OIA-CRB}) & Establishment of Self-sustaining Biological Control of Coconut Rhinoceros Beetle Biotype G in Micronesia & \$239,994 \\
		\midrule
		FS-CRB (\ref{FS-CRB})& Establishment of Self-sustaining Biological Control of Coconut Rhinoceros Beetle Biotype G in Micronesia &  \$98,240 \\
		\midrule
		BIODIVERSITY (\ref{BIODIVERSITY}) &                                                                       Guam Forest Biodiversity Inventory &  \$80,000 \\
		\midrule		
		8SPOT (\ref{8SPOT}) &                                                                 Captive Breeding of Eight-spot Butterfly &  \$23,212 \\
		\midrule
		FS-CRB-HR (\ref{FS-CRB-HR})&                         Improving Coconut Rhinoceros Beetle Breeding Site Detection Using Harmonic Radar &  \$23,000 \\
		\midrule
		WPDN2 (\ref{WPDN2}) &                                                                  Western Plant Diagnostic Network FY2022 &  \$15,000 \\
		\midrule
		WPDN3 (\ref{WPDN2}) &                                                                  Western Plant Diagnostic Network FY2022 &  \$15,000 \\
		\bottomrule
	\end{tabular}
\end{table}

\begin{figure}[h!]
	\label{gannt}
	\begin{ganttchart}[time slot format=isodate, expand chart=\textwidth]{2020-01-01}{2023-12-31}
		%	\gantttitle{2011}{12} \\
		\gantttitlecalendar{year} \\
		\ganttgroup{reporting period}{2022-06-26}{2023-06-26} \\
		\ganttbar{DOI-OIA}{2020-06-15}{2023-09-30} \\
		\ganttbar{McStennis}{2020-05-14}{2022-09-30}\\
		\ganttbar{8spot}{2020-01-01}{2022-09-30}\\
		\ganttbar{WPDN2}{2021-09-01}{2022-08-31}\\
		\ganttbar{WPDN3}{2022-09-01}{2023-08-31}\\		
		\ganttbar{FS-HR}{2020-06-17}{2022-12-31}\\
		\ganttbar{FS-CRB}{2020-06-17}{2023-12-01}\\
		\ganttvrule{}{2020-06-26}
		\ganttvrule{}{2021-06-26}
		\ganttvrule{}{2022-06-26}
		\ganttvrule{}{2023-06-26}
	\end{ganttchart}
	\caption{Performance periods for grants which were active during the reporting period (2022-06-26 through 2023-06-26).}
\end{figure}

\clearpage
\subsection{OIA-CRB Biological Control of Coconut Rhinoceros Beetle Biotype-G in Micronesia \$240K}
\label{OIA-CRB}

\subsubsection{Key data}
\begin{itemize}
	\setlength\itemsep{0em}	
	\item \textbf{Code:} OIA-CRB
	\item \textbf{Title:} Establishment of Self-sustaining Biological Control of Coconut Rhinoceros Beetle Biotype G in Micronesia
	\item \textbf{Start date:} 2020-05-14
	\item \textbf{End date:} 2023-09-30
	\item \textbf{Total budget:} \$239,994
	\item \textbf{Federal ID:} D20AP00060
	\item \textbf{UOG ID:} DOI Biocontrol CRB
	\item \textbf{UOG Account:} 30-2F-311150
	\item \href{https://github.com/aubreymoore/2020-DOI-CRB-Biocontrol}{GitHub repository}
\end{itemize}

\subsubsection{Documents}
\begin{itemize}
	\setlength\itemsep{0em}	
	\item \href{https://github.com/aubreymoore/2020-DOI-CRB-Biocontrol/blob/master/doi_proposal.pdf}{Proposal}
	\item \href{https://github.com/aubreymoore/2020-DOI-CRB-Biocontrol/blob/master/D20AP00060-Grant\%20Award\%20Document.pdf}{Award letter}
	\item \href{https://github.com/aubreymoore/2020-DOI-CRB-Biocontrol/raw/master/Reporting\%20requirements\%20\%20D17AP00107.pdf}{Reporting requirements}
	\item \href{https://github.com/aubreymoore/2020-DOI-CRB-Biocontrol/raw/master/doi_report1.pdf}{Report 1}
\end{itemize}

\newpage
\subsection{BIODIVERSITY Guam Forest Biodiversity Inventory \$80K}
\label{BIODIVERSITY}
\subsubsection{Key data}
\begin{itemize}
	\setlength\itemsep{0em}
	\item \textbf{Code:} BIODIVERSITY
	\item \textbf{Title:} Guam Forest Biodiversity Inventory
	\item \textbf{Funding source:} McIntire-Stennis (administered by CNAS)
	\item \textbf{Reporting system:} \href{https://portal.nifa.usda.gov/portal/front/login}{REEport}
	\item \textbf{Start date:} 2018-10-15
	\item \textbf{End date:} 2022-09-30
	\item \textbf{Total budget:} \$16,000 per year for each of 4 years
	\item \textbf{Federal ID:} GUA0930
	\item \textbf{UOG ID:}
	\item \textbf{UOG Account:}
	\item \href{https://github.com/aubreymoore/McIntire-Stennis}{GitHub repository}
\end{itemize}

\subsubsection{Documents}
\begin{itemize}
	\setlength\itemsep{0em}	
	\item \href{https://github.com/aubreymoore/McIntire-Stennis/raw/master/MS_Project_Proposal_2018/ms_proposal_2018.pdf}{2018-06-21 Proposal}
	\item \href{https://github.com/aubreymoore/McIntire-Stennis/raw/master/project_initiation.pdf}{2018-10-08 Project initiation}
	\item \href{https://github.com/aubreymoore/McIntire-Stennis/raw/master/2019-report.pdf}{2020-01-02 2019 Annual report}
	\item \href{https://github.com/aubreymoore/McIntire-Stennis/raw/master/2020\%20Annual\%20Report/2020\%20McS\%20Annual\%20Report.pdf}{2020-12-28 2020 Annual report}
	\item \href{https://github.com/aubreymoore/McIntire-Stennis/raw/master/2021\%20Annual\%20Report/submitted_for_review.pdf}{2021-12-18 2021 Annual report}
	\item Final report due 2022-12-31.
\end{itemize}





\newpage
\subsection{8SPOT Eight Spot Butterfly Conservation \$20K}
\label{8SPOT}
\subsubsection{Key data}
\begin{itemize}
	\setlength\itemsep{0em}
	\item \textbf{Code:} 8SPOT
	\item \textbf{Title:} Captive Breeding of Eight-spot Butterfly
	\item \textbf{Start date:} 2013-10-01
	\item \textbf{End date:} 2022-09-30
	\item \textbf{Total budget:} \$23,212
	\item \textbf{Funding Agency:} DOI-FWS (via GDOA-DAWR)
	\item \textbf{Federal ID (FAIN):} F13AF01300
	\item \textbf{UOG ID:}
	\item \textbf{UOG Account:} 30-1F-315058-R
	\item \href{https://github.com/aubreymoore/Hypollimnas-octocula-conservation}{GitHub repository}
\end{itemize}

\subsubsection{Documents}
\begin{itemize}
	\setlength\itemsep{0em}	
	\item \href{https://github.com/aubreymoore/Hypollimnas-octocula-conservation/blob/master/8spot-award-notice.pdf}{Award letter (includes scope of work and budget)}
	\item \href{https://github.com/aubreymoore/Hypollimnas-octocula-conservation/blob/master/8spot-award-notice-updated.pdf}{Updated Award Letter}
\end{itemize}




\newpage
\subsection{WPDN2 Western Plant Diagnostic Network FY2022 \$15K}
\label{WPDN2}

\subsubsection{Key data}
\begin{itemize}
	\setlength\itemsep{0em}	
	\item \textbf{Code:} WPDN2
	\item \textbf{Title:} Western Plant Diagnostic Network FY2022
	\item \textbf{Start date:} 2021-09-01
	\item \textbf{End date:} 2022-08-31
	\item \textbf{Total budget:} \$15,000
	\item \textbf{UOG ID:} WPTRC-UCDAVIS/USDA WPLANTDI
	\item \textbf{UOG Account:} 61-1F-243432
	\item \href{https://github.com/aubreymoore/WPDN}{GitHub repository}
\end{itemize}

\subsubsection{Documents}
\begin{itemize}
	\setlength\itemsep{0em}	
	\item \href{https://github.com/aubreymoore/WPDN/blob/main/WPDN\%202021-2022\%20workplan\%20and\%20budget.pdf}{Work plan and budget}
	\item \href{https://github.com/aubreymoore/WPDN/raw/main/FY2022/WPDN-FY2022-Award-Letter.pdf}{Award letter}
	\item \href{https://github.com/aubreymoore/WPDN/raw/main/FY2022/UOG-account-setup.pdf}{UOG account setup}
\end{itemize}


\newpage
\subsection{WPDN3 Western Plant Diagnostic Network FY2022 \$15K}
\label{WPDN3}

\subsubsection{Key data}
\begin{itemize}
	\setlength\itemsep{0em}	
	\item \textbf{Code:} WPDN3
	\item \textbf{Title:} Western Plant Diagnostic Network FY2023
	\item \textbf{Start date:} 2022-09-01
	\item \textbf{End date:} 2023-08-31
	\item \textbf{Total budget:} \$15,000
	\item \textbf{UOG ID:} WPTRC-UCDAVIS/USDA WPLANTDI
	\item \textbf{UOG Account:} 
	\item \href{https://github.com/aubreymoore/WPDN}{GitHub repository}
\end{itemize}

%\subsubsection{Documents}
%\begin{itemize}
%	\setlength\itemsep{0em}	
%	\item \href{https://github.com/aubreymoore/WPDN/blob/main/WPDN\%202021-2022\%20workplan\%20and\%20budget.pdf}{Work plan and budget}
%	\item \href{https://github.com/aubreymoore/WPDN/raw/main/FY2022/WPDN-FY2022-Award-Letter.pdf}{Award letter}
%	\item \href{https://github.com/aubreymoore/WPDN/raw/main/FY2022/UOG-account-setup.pdf}{UOG account setup}
%\end{itemize}
%


\newpage
\subsection{FS-CRB-HR Harmonic Radar \$23K}
\label{FS-CRB-HR}

\subsubsection{Key data}
\begin{itemize}
	\setlength\itemsep{0em}
	\item \textbf{Code:} FS-CRB-HR
	\item \textbf{Long title:} Improving Coconut Rhinoceros Beetle Breeding Site Detection Using Harmonic Radar
	\item \textbf{Start date:} 2020-06-17
	\item \textbf{End date:} 2022-12-31
	\item \textbf{Total budget:} \$23,000
	\item \textbf{Federal ID:} 20-DG-11052021-227
	\item \textbf{UOG ID:} CNAS-USDA-CRB Harmonic Radar
	\item \textbf{UG Account:} 30-2F-311144-R
	\item \href{https://github.com/aubreymoore/Harmonic-Radar}{GitHub repository}
\end{itemize}

\subsubsection{Documents}
\begin{itemize}
	\setlength\itemsep{0em}		
	\item \href{https://github.com/aubreymoore/Harmonic-Radar/blob/master/USFS-harmonic-radar-proposal.pdf}{Proposal}
	\item \href{https://github.com/aubreymoore/grant-tracker/blob/main/mydocs/20-DG-227-UOG-CRB-Harmonic-Radar-Fully-Signed-Grant.pdf}{Award letter}
	\item \href{https://github.com/aubreymoore/grant-tracker/blob/main/mydocs/20-DG-227\%20UOG\%20Mod\%201\%20Fully\%20Signed.pdf}{Extension until 2022-12-31}
	\item \href{https://github.com/aubreymoore/Harmonic-Radar/raw/master/FS-CRB-HR-report1.pdf}{Report 1 (2021-01-31)}
	\item \href{}{Report 2 (2021-07-31)}
	\item \href{}{Final report (90 days after expiration date)}
\end{itemize}





\newpage
\subsection{FS-CRB CRB Biocontrol \$98K}
\label{FS-CRB}

\subsubsection{Key data}
\begin{itemize}
	\setlength\itemsep{0em}	
	\item \textbf{Code:} FS-CRB
	\item \textbf{Long title:} Establishment of Self-sustaining Biological Control of Coconut Rhinoceros Beetle Biotype G in Micronesia
	\item \textbf{Start date:} 2020-06-17
	\item \textbf{End date:} 2022-12-31
	\item \textbf{Total budget:} \$98,240
	\item \textbf{Federal ID:} 20-DG-11052021-229
	\item \textbf{UOG ID:} CNAS-USDA Control of CRB
	\item \textbf{UG Account:} 30-2F-311143-R
	\item \href{https://github.com/aubreymoore/2020-FS-CRB-biocontrol-project}{GitHub repository}
\end{itemize}

\subsubsection{Documents}
\begin{itemize}
	\setlength\itemsep{0em}	
	\item \href{https://github.com/aubreymoore/2020-FS-CRB-biocontrol-project/blob/master/combined-proposal.pdf}{Proposal}
	\item \href{https://github.com/aubreymoore/grant-tracker/blob/main/mydocs/20-DG-229\%20UOG\%20LFA\%20CRB\%20Fully\%20Signed\%20Grant.pdf}{Award letter}
	\item \href{https://github.com/aubreymoore/grant-tracker/blob/main/mydocs/20-DG-229\%20UOG\%20Mod\%201\%20Fully\%20Signed.pdf}{Extension until 2022-12-31}
	\item \href{https://github.com/aubreymoore/2020-FS-CRB-biocontrol-project/raw/master/FS-CRB-biocontrol-report1.pdf}{Report 1 (2021-01-31)}
	\item \href{}{Report 2 (2021-07-31)}
	\item \href{}{Final report}
	\item \href{https://github.com/aubreymoore/2020-FS-CRB-biocontrol-project/blob/master/20-DG-229-UOG-Mod-2-Fully-Signed.pdf}{2021-06-15 Amended agreement}
\end{itemize}





%\newpage
%\section{Submitted Grant Proposals (n=1)}
%
%\subsection{WPDN2 Western Plant Diagnostic Network FY2023-FY2026 \$60K}
%\label{WPDN4YR}
%
%\subsubsection{Key data}
%\begin{itemize}
%	\setlength\itemsep{0em}	
%	\item \textbf{Code:} WPDN-4YR
%	\item \textbf{Title:} Western Plant Diagnostic Network FY2022-FY2026
%	\item \textbf{Start date:}
%	\item \textbf{End date:}
%	\item \textbf{Total budget:} \$60,000
%	\item \textbf{UOG ID:}
%	\item \textbf{UOG Account:}
%	\item \href{https://github.com/aubreymoore/WPDN/tree/main/4year}{GitHub repository}
%\end{itemize}
%
%\subsubsection{Documents}
%\begin{itemize}
%	\setlength\itemsep{0em}	
%	\item \href{https://github.com/aubreymoore/WPDN/raw/main/4year/WPDN\%20FY23-FY26\%20workplan\%20and\%20budget\%20for\%20UOG.pdf}{Work plan and budget}
%\end{itemize}
%














\newpage
\section{Part IV (Work Plan for 15JUN2022-14JUN2023): Extension and Community Activities (ECA) 51\%}

\subsection{ECA-PA1 Insect Diagnostic Services} 
I plan to continue providing insect diagnostic services.

\subsection{ECA-PA2 Detection and Documentation of Invasive Species} 
I will continue to document new island records of insects detected in Micronesia.

The International Union for Conservation of Nature (IUCN-ISSG) is
building a Global Register of Introduced and Invasive Species. I have
volunteered to coordinate building a check list for species on Guam.

The Guam Invasive Species Council is required to maintain a list on
invasive species on Guam. I have volunteered to be ``registrar''
for this list.

\subsection{ECA-PA3 University of Guam Insect Collection}
With the limited space problem partially solved, I intend to re-established the UOG insect collection internship to train entomology students how to curate an institutional insect collection.
The current focus will be on adding specimen images to the online database.

\subsection{ECA-PA4 Mitigation of Damage to Guam's Ecosystems by Invasive Species} 
I plan to continue providing control recommendations for invasive insect species when control methods are available.

I will continue with applied research on CAS and CRB in an effort to mitigate the major damage being done by these pests.

\subsection{ECA-PA5 National Plant Diagnostic Network (NPDN)}
I will continue to act as UOG coordinator for NPDN.

\subsection{ECA-PA6 Guam Invasive Species Advisory Committee (GISAC) and Guam Invasive Species Council (GISC)}
I plan to continue as an active member of GISAC and GISC.

I plan to participate in a review of the Guam Invasive Species Management Plan.

\subsection{ECA-PA7 Public Outreach: Internet}
I will continue to create and maintain websites about insects in Micronesia.

\subsection{ECA-PA8 Public Outreach: Presentations}
I will continue to create and make presentations.

\subsection{ECA-PA10 Public Outreach: Public GitHub Repositories}
I will continue to create and maintain GitHub repositories to share my work.


\newpage
\section{Part IV (Work Plan for 15JUN2022-14JUN2023): Creative/Scholarly Activities or Research (CSR) 34\%}

\subsection{CSR-PA1 Peer Reviewed Publications} 
I plan to publish at least 1 peer reviewed journal article as author or co-author.

\subsection{CSR-PA4 Coconut Rhinoceros Beetle (CRB) Biocontrol}
I will continue applied research with the goal of finding a solution to the CRB problem on Guam and elsewhere in Micronesia.

I have been asked by the US Forestry Service to write a \href{https://www.fs.fed.us/foresthealth/publications/fidls/index.shtml}{Forest Insect and Disease Leaflet (FIDL)} for coconut rhinoceros beetle and I will work on this. 

I plan to participate in the \href{https://www.pacificrisa.org/pesc/}{First Pacific Ecological Security Conference (PESC) in Palau}. I am a member of the planning committee for development of a strategic action plan for coconut rhinoceros beetle, one of three goals for this meeting.

I plan to participate in the Annual Meeting of the Entomological Society of America in November 2022.

\subsection{CSR-PA5 Guam Biodiversity Inventory} I am working to complete this project before the grant expires on 2022-09-30 and to complete a final report due 2022-12-31.

\subsection{CSR-PA6 Cycad Aulacaspis Scale (CAS) Biocontrol}

I will work to help find funding to implement Dr. Cave's recommendations. 

\subsection{CSR-PA7 Eight Spot Butterfly (ESB) Conservation}

I will work to complete the USFWS grant which supports this work and to write a final report. The objective is to test the feasibility of captive rearing for eight-spot butterflies. This work is being done in collaboration with Dr. George Fiedler from the Biology Department and Matt Putnam, a graduate student from the Center for Island Sustainability. 

\newpage
\section{Part IV (Work Plan for 15JUN2022-14JUN2023): University and Community Service (UCS) 15\%}

\subsection{UCS-PA2 Graduate Instruction}As member of the Environmental Science graduate faculty, I will continue to serve on graduate committees and present guest lectures.

\subsection{UCS-PA3 Faculty Committees}
I will continue to serve on faculty committees. 

\end{document}
